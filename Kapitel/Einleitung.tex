\section{Einleitung}
\label{sec.Einleitung}

Dieser Bericht befasst sich mit der Modellierung und der Regelung eines invertierten Pendels, das auch \emph{Furuta-Pendel} genannt wird. 
Ziel des Labors ist es, einen robusten Regler zu entwerfen der die folgenden drei Aufgaben erfüllt:
\begin{enumerate}
	\item Das Pendel aufschwingen, bis der \emph{Schwingarm} durch die grenzstabile aufrechte Lage schwingt.
	\item Den \emph{Schwingarm} in der aufrechten Lage fangen und dort halten.
	\item Den \emph{Dreharm} zurück zur Ausgangslage bringen, ohne dass dabei der \emph{Schwingarm} aus der aufrechten Lage herausfällt.
\end{enumerate}
Im Rest dieses Berichtes wird der \emph{Dreharm} mit Arm~1 und der \emph{Schwingarm} mit Arm~2 bezeichnet.
Für weitere Erläuterungen sei auf Abschnitt~\ref{sec.Theorie} verwiesen.
Der Bericht ist Teil des Fachlabors Mechatronik im Sommersemester 2015 an der Technischen Universität Hamburg-Harburg und stellt die im Zuge dieses Labors durchgeführten Berechnungen, Modellierungen und versuche dar. 
Für die mathematische Modellierung und die Auslegung der Regler werden Matlab und Simulink verwendet.
Diese Programme stehen den Autoren von der Universität aus zur Verfügung.

Abschnitt~\ref{sec.Theorie} befasst sich mit den nötigen theoretischen Grundlagen und der mathematischen Modellierung des Pendels.
Dort wird primär auf die Quelle~\citep{Cazzolato.2011} eingegangen, welche als Vorlage genutzt wurde und vom Betreuer des Labors zur Verfügung gestellt wurde.

In Abschnitt~\ref{sec.Parameter} werden die für die Regelung nötigen Parameter aus dem realen Pendel abgeleitet. 
Gemessene Größen wie Massen und Längen des realen Pendels werden dazu umgeformt und zusammengefasst um diese auf die Form des mathematischen Modells zu bringen.
Es folgt eine Auflistung der auf diese Weise identifizierten Größen, die in das mathematische Modell einfließen und die Grundlage für die Regelung bilden.

Die Auslegung und Implementierung der Regler in Matlab und Simulink wird in Abschnitt~\ref{sec.Controller} behandelt.
Es wird auf die verschiedenen Formen von Regelungsalgorithmen eingegangen die eingesetzt werden um die oben beschriebenen Anforderungen zu erfüllen und wie zwischen diesen je nach Zustand des Pendels gewechselt werden kann.
Außerdem wird ein Algorithmus zur Unterscheidung zweier unterschiedlicher Pendelarme ausgelegt. 
Es soll damit möglich sein, automatisch zu identifizieren welcher der beiden vorgegebenen Pendelarme aktuell montiert ist. 
Die Pendelarme unterscheiden sich in Masse und Stablänge.

Ergebnisse verschiedener Messungen an dem realen Furuta-Pendel werden in Abschnitt~\ref{sec.Ergebnisse} präsentiert. 
Dort finden sich verschiedene Versuche zum Testen der Funktion und der Robustheit des Reglers.
Es werden jeweils Messergebnisse präsentiert, welche die Funktionalität der verschiedenen Regelungsalgorithmen und deren Zusammenspiel verdeutlichen sollen.

Im letzten Abschnitt folgt eine kurze Zusammenfassung, in der auf die wesentlichen Erkenntnisse dieses Laborprojekts noch einmal eingegangen wird.
Außerdem wird hier auf mögliche Erweiterungen für die Regelung eines Furuta-Pendels eingegangen.

