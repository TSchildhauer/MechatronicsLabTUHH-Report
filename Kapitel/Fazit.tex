\section{Fazit}
Zusammenfassend wurde das Ziel des Fachlabors Mechatronik - das Aufschwingen und Halten eines inversen Pendels - erreicht, was in diesem Bericht erläutert wurde. Es wurde zunächst ein (vereinfachtes) Modell hergeleitet, auf dessen Basis die Implementierung der Regler stattfand.\\

Durch die Verwendung eines nichtlinearen Swing-Up-Controllers konnte das Aufschwingen beschleunigt werden, wie in Kapitel \ref{sec.Ergebnisse} bewiesen. Das Verhalten wurde sogar durch die Erweiterung durch einen Bang-Bang-Controller verbessert. Zum Fangen und Stabilisieren des Pendels nahe des oberen Equilibriums wurde eine lineare, auf einem Observer basierende LQ-Regelung implementiert. Durch die Trennung in Catcher- und Stabilizer-Regler konnte auch $\theta_1$ auf den Ursprung zurückgeholt werden.\\
Das Verhalten wurde auch durch unterschiedlich große Störimpulse nachvollzogen, wobei der Regler bei nahezu allen Störungen funktionierte. Allerdings trat vereinzelt bei sehr starken Störimpulsen eine Instabilität des Systems auf. Um ein solches Verhalten zu verhindern, könnte man im nächsten Schritt einen Watchdog implementieren, der Anhand der Winkelgeschwindigkeit $\dot{\theta}_1$ erkennt, wenn das Pendel nicht mehr erfolgreich geregelt werden kann. Liegt der Wert von $\dot{\theta}_1$ für eine bestimmte Zeit nicht in dem Intervall $[-\epsilon,+\epsilon]$, so kann der Watchdog einen kurzzeitigen Stopp des Pendels veranlassen und es anschließend neu starten.\\

Darüber hinaus könnte eine bessere Modellbildung z.B. der Reibungskräften und Verformungen des Pendels sowie die Beachtung der Umwelteinflüsse, wie z.B. Temperatur und Wind, zu einer besseren Regelbarkeit führen. \\

Bisher wurden zudem auch nur 2 Parametersätze zur automatischen Erkennung und Regelung verschiedener Pendellängen und -massen eingepflegt. Dies könnte noch erweitert werden, um die Robustheit des Systems zu erhöhen. 
