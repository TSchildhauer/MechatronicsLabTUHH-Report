\section{Theorie}
\label{sec.Theorie}
In diesem Abschnitt wird auf die Theorie des sogenannten \emph{Furuta Pendels} eingegangen. Zunächst wird ein Modell des Pendels beschrieben. Im Anschluss wird die Theorie dieses Modells auf ein echtes Furuta Pendel angewandt. %Thomas: Diesen Text überarbeiten wenn der Abschnitt fertig ist.

\subsection{Furuta Pendel}
\label{sub.Furuta-Pendel}
Bei dem \emph{Furuta Pendel}, auch ?drehbares??? invertiertes Pendel genannt, handelt es sich um ein 1992 von Katsuhisa Furuta entwickeltes mechanisches Pendel. 
Es besteht aus einem angetriebenen arm, welcher in der horizontalen Ebene rotieren kann. 
An diesem Arm ist ein Pendelarm befestigt, welcher frei in der zum angetriebenen Arm orthogonalen Ebene rotieren kann. 
Das Furuta Pendel ist ein Beispiel eines komplexen nichtlinearen Oszillators. \citet{Cazzolato.2011}

%Thomas: Zusätzlich hier eine Foto des pendels einbinden?

\begin{figure}[htbp]
	\label{fig.furuta-schematic}
	\centering
	\includegraphics[width=1.\textwidth]{Grafiken/furuta3.jpg}
	\caption{Schema des Furuta Pendels nach~\cite{Cazzolato.2011}. }
\end{figure}

Abbildung~\ref{fig.furuta-schematic} zeigt das Schema eines Furuta Pendels, welches von einem DC-Elektromotor angetrieben wird.
Mit dem Motor wird ein Drehmoment $\tau_1$ auf Arm~1 des Pendels gebracht. %??? Thomas: gebracht~
Das Drehmoment $\tau_2$ bezeichnet das Kopplungsmoment zwischen Arm~1 und Arm~2. %Thomas: kpplung~
Die Verbindung zwischen Arm~1 und Arm~2 rotiert frei und ist nicht angetrieben.
Die beiden Arme haben Längen $L_1$ und $L_2$ sowie Massen $m_1$ und $m_2$.
$l_1$ und $l_2$ bezeichnen die jeweiligen Abstände der Massenmittelpunkte zum Rotationspunkt der Arme.
Die Arme haben Trägheitstensoren $J_1$ und $J_2$ um ihren jeweiligen Massenmittelpunkt und ihre Gelenke unterliegen linearer Dämpfung mit Dämpfungskonstanten $b_1$ und $b_2$.

\subsubsection{Annahmen}
\label{sub.sub.Annahmen}
Vor der Aufstellung des Gleichungssystems werden in~\cite{Cazzolato.2011} einige Annahmen getroffen, die an dieser Stelle kurz wiederholt werden:
\begin{list}{}{}
\item{(i)} Die Kopplung zwischen Motorachse und Arm~1 wird als ideal starr angenommen.
\item{(ii)} Die Motorachse und beide Pendelarme werden als ideal starr angenommen.
\item{(iii)} Die Koordinatensysteme von Arm~1 und Arm~2 sind die Hauptachsensysteme, sodass die Trägheitstensoren diagonal sind.
\item{(iv)} Die Trägheit des Motors wird vernachlässigt. %Thomas: trifft das auf uns zu????
\item{(v)} Es wird von reiner linearer (viskoser) Dämpfung ausgegangen.
\end{list}

%Thomas... Torque und so weiter und energien
%Thomas: welche der folgenden müssen nummern erhalten?
\begin{equation}
 E_{p1} = 0
\end{equation}

\begin{equation}
E_{k1}=\frac{1}{2}(v^T_{1c}m_1v_{1c}+\omega^T_1J_1\omega_1)=\frac{1}{2}\dot{\theta}^2_1(m_1l_1+J_{1zz})
\end{equation}

\begin{equation}
E_{p2}=gm_2l_2(\cos(\theta_2)-1)
\end{equation}

\begin{eqnarray}
E_{k2}&=&\frac{1}{2}(v^T_{2c}m_2v_{2c}+\omega^T_2J_2\omega_2)\nonumber \\
&=&\frac{1}{2}\dot{\theta}^2_1(m_2L^2_2+(m_2l^2_2+J_{2yy})\sin^2(\theta_2)+J_{2xx}\cos^2(\theta_2))\nonumber \\
&&+\frac{1}{2}\dot{\theta}^2_2(J_{2zz}+m_2l^2_2)+m_2L_1l_2\cos(\theta_2)\dot{\theta}_1\dot{\theta}_2
\end{eqnarray}

\begin{equation}
L=E_k-E_p
\end{equation}

\begin{equation}
\frac{d}{dt}(\frac{\partial L}{\partial\dot{q_i}})+b_i\dot{q}_i-\frac{\partial L}{\partial q_i}=Q_i
\end{equation}

\begin{equation}
\begin{bmatrix}
\tau_1 \\
\tau_2
\end{bmatrix}
=
\begin{bmatrix}
\begin{pmatrix}
\ddot{\theta}_1(J_{1zz}+m_1l^2_1+m_2L^2_1+(J_{2yy}+m_2l^2_2) 						\\
\times \sin^2(\theta_2)+J_{2xx}cos^2(\theta_2))+\ddot{\theta}_2m_2L_1l_2\cos(\theta_2)			\\
-m_2L_1l_2\sin(\theta_2)\dot{\theta}^2_2+\dot{\theta}_1\dot{\theta}_2\sin(2\theta_2)	\\
\times(m_2l^2_2+J_{2yy}-J_{2xx})+b_1\dot{\theta}_1
\end{pmatrix}
\\
\begin{pmatrix}
\ddot{\theta}_1m_2L_1l_2\cos(\theta_2)+\ddot{\theta}_2(m_2l^2_2+J_{2zz})	\\
+\frac{1}{2}\dot{\theta}_1\sin(2\theta_2)(-m_2l^2_2-J_{2yy}+J_{2xx})						\\
+b_2\dot{\theta}_2+gm_2l_2\sin(\theta_2)
\end{pmatrix}
\end{bmatrix}
\end{equation}

\begin{eqnarray}
J_1=
\begin{bmatrix}
J_{1xx} & 0 & 0\\
0 & J_{1yy} & 0\\
0 & 0 & J_{1zz}
\end{bmatrix}
\approx
\begin{bmatrix}
0 & 0 & 0\\
0 & J_{1} & 0\\
0 & 0 & J_{1}
\end{bmatrix}
\nonumber \\
J_2=
\begin{bmatrix}
J_{2xx} & 0 & 0\\
0 & J_{2yy} & 0\\
0 & 0 & J_{2zz}
\end{bmatrix}
\approx
\begin{bmatrix}
0 & 0 & 0\\
0 & J_{2} & 0\\
0 & 0 & J_{2}
\end{bmatrix}
\end{eqnarray}

\begin{eqnarray}
\hat{J}_1 &=& J_1 + m_1l^2_1	\nonumber	\\
\hat{J}_2 &=& J_2 + m_2l^2_2	\nonumber	\\
\hat{J}_0 &=& J_1 + m_1l^2_1 + m_2L^2_1
\end{eqnarray}

\begin{equation}
\begin{bmatrix}
\tau_1 \\
\tau_2
\end{bmatrix}
=
\begin{bmatrix}
\begin{pmatrix}
\ddot{\theta}_1(\hat{J}_0+\hat{J}_2\sin^2(\theta_2))+\ddot{\theta}_2m_2L_1l_2\cos(\theta_2)			\\
-m_2L_1l_2\sin(\theta_2)\dot{\theta}^2_2+\dot{\theta}_1\dot{\theta}_2\hat{J}_2\sin(2\theta_2)+b_1\dot{\theta}_1
\end{pmatrix}
\\
\begin{pmatrix}
\ddot{\theta}_1m_2L_1l_2\cos(\theta_2)+\ddot{\theta}_2\hat{J}_2-\frac{1}{2}\dot{\theta}^2_1\hat{J}_2\sin(2\theta_2)						\\
+b_2\dot{\theta}_2+gm_2l_2\sin(\theta_2)
\end{pmatrix}
\end{bmatrix}
\end{equation}


\colorbox{yellow}{Die folgenden Gleichungen enthalten noch den Fehler, den wir finden sollten.} \\
\colorbox{yellow}{Ich habe die Lösung gerade nicht parat.}
\begin{equation}
\ddot{\theta}_1 =
\frac{
\begin{bmatrix}
-\hat{J}_2b_1 \\ 
m_2L_1l_2\cos(\theta_2)b_2 \\ 
-\hat{J}^2_2\sin(2\theta_2) \\ 
-\frac{1}{2}\hat{J}_2m_2L_1l_2\cos(\theta_2)\sin(2\theta_2) \\ 
\hat{J}_2m_2L_1l_2\sin(\theta_2)
\end{bmatrix}^T
\begin{bmatrix}
\dot{\theta}_1 \\ 
\dot{\theta}_2 \\ 
\dot{\theta}_1\dot{\theta}_2 \\ 
\dot{\theta}^2_1 \\ 
\dot{\theta}^2_2
\end{bmatrix}
+
\begin{bmatrix}
\hat{J}_2 \\ 
-m_2L_1l_2\cos(\theta_2) \\ 
\frac{1}{2}m^2_2l^2_2L_1\sin(2\theta_2)
\end{bmatrix}^T
\begin{bmatrix}
\tau_1 \\ 
\tau_2 \\ 
g
\end{bmatrix} }
{\hat{J}_0\hat{J}_2+\hat{J}^2_2\sin^2(\theta_2)-m^2_2L^2_1l^2_2\cos^2(\theta_2)}
\end{equation}



\begin{equation}
\ddot{\theta}_2 =
\frac{
\begin{bmatrix}
m_2L_1l_2\cos(\theta_2)b_1 \\ 
-b_2(\hat{J}_0+\hat{J}_2\sin^2(\theta_2)) \\ 
m_2L_1l_2\hat{J}_2\cos(\theta_2)\sin(2\theta_2) \\ 
-\frac{1}{2}\sin(2\theta_2)(\hat{J}_0\hat{J}_2+\hat{J}^2_2\sin^2(\theta_2)) \\ 
-\frac{1}{2}m^2_2L^2_1l^2_2\sin(2\theta_2)
\end{bmatrix}
\begin{bmatrix}
\dot{\theta}_1 \\ 
\dot{\theta}_2 \\ 
\dot{\theta}_1\dot{\theta}_2 \\ 
\dot{\theta}^2_1 \\ 
\dot{\theta}^2_2
\end{bmatrix}
+
\begin{bmatrix}
-m_2l_2\cos(\theta_2) \\ 
\hat{J}_0+\hat{J}_2\sin^2(\theta_2) \\ 
-m_2l_2\sin(\theta_2)(\hat{J}_0+\hat{J}_2\sin^2(\theta_2))
\end{bmatrix}
\begin{bmatrix}
\tau_1 \\ 
\tau_2 \\ 
g
\end{bmatrix} }
{\hat{J}_0\hat{J}_2+\hat{J}^2_2\sin^2(\theta_2)-m^2_2L^2_1l^2_2\cos^2(\theta_2)}
\end{equation}

\begin{eqnarray}
\theta_{1e} &=& 0		\nonumber \\
\theta_{2e} &=& \pi	\nonumber \\
\dot{\theta}_{1e} &=& 0		\nonumber \\
\dot{\theta}_{2e} &=& 0	
\end{eqnarray}

\begin{equation}
\begin{bmatrix}
\dot{\theta}_1 \\ 
\dot{\theta}_2 \\ 
\ddot{\theta}_1 \\ 
\ddot{\theta}_2
\end{bmatrix}
=
\begin{bmatrix}
0 & 0 & 1 & 0 \\ 
0 & 0 & 0 & 1 \\ 
A_{31} & A_{32} & A_{33} & A_{34} \\ 
A_{41} & A_{42} & A_{43} & A_{44}
\end{bmatrix}
\begin{bmatrix}
\theta_1 \\ 
\theta_2 \\ 
\dot{\theta}_1 \\
\dot{\theta}_2
\end{bmatrix}
+
\begin{bmatrix}
0 & 0 \\ 
0 & 0 \\ 
B_{31} & B_{32} \\ 
B_{41} & B_{42}
\end{bmatrix}
\begin{bmatrix}
\tau_1 \\ 
\tau_2
\end{bmatrix}
\end{equation}

\begin{eqnarray}
A_{31} &=& 0	\nonumber \\
A_{32} &=& \frac{gm^2_2l^2_2L_1}{(\hat{J}_0\hat{J}_2-m^2_2L^2_1l^2_2)}	\nonumber \\
A_{33} &=& \frac{-b_1\hat{J}_2}{(\hat{J}_0\hat{J}_2-m^2_2L^2_1l^2_2)}	\nonumber \\
A_{34} &=& \frac{-b_2m_2l_2L_1}{(\hat{J}_0\hat{J}_2-m^2_2L^2_1l^2_2)}	\nonumber \\
A_{41} &=& 0	\nonumber \\
A_{42} &=& \frac{gm_2l_2\hat{J}_0}{(\hat{J}_0\hat{J}_2-m^2_2L^2_1l^2_2)}	\nonumber \\
A_{43} &=& \frac{-b_1m_2l_2L_1}{(\hat{J}_0\hat{J}_2-m^2_2L^2_1l^2_2)}	\nonumber \\
A_{44} &=& \frac{-b_2\hat{J}_0}{(\hat{J}_0\hat{J}_2-m^2_2L^2_1l^2_2)}	\nonumber \\
B_{31} &=& \frac{\hat{J}_2}{(\hat{J}_0\hat{J}_2-m^2_2L^2_1l^2_2)}	\nonumber \\
B_{41} &=& \frac{m_2L_1l_2}{(\hat{J}_0\hat{J}_2-m^2_2L^2_1l^2_2)}	\nonumber \\
B_{32} &=& \frac{m_2L_1l_2}{(\hat{J}_0\hat{J}_2-m^2_2L^2_1l^2_2)}	\nonumber \\
B_{42} &=& \frac{\hat{J}_0}{(\hat{J}_0\hat{J}_2-m^2_2L^2_1l^2_2)}	\nonumber \\
\end{eqnarray}

\begin{equation}
\tau=K_mi
\end{equation}
Herleitung der Parameter (1 zu 1 aus dem Dokument übernommen.. die Herangehensweise muss also noch verändert werden!):
\begin{eqnarray}
L_1 &=& l_a \nonumber \\
L_2 &=& l_{pm}
\end{eqnarray}
\begin{eqnarray}
m_1 &=& m_b+m_r+m_{sens} \nonumber \\
m_2 &=& m_p+m_{pm}
\end{eqnarray}

\begin{equation}
J_{arm} = m_r \frac{l^2_r}{12}+m_r(\frac{1}{2}l_r+l_b-l_a)^2
\end{equation}

\begin{equation}
J_{pend1}=m_bl^2_a
\end{equation}

\begin{equation}
J_{sens}=m_{sens}(l_a-l_b-l_r-\frac{1}{2}l_c)^2
\end{equation}

\begin{equation}
J_{ps}=m_p(r_b+\frac{1}{2}l_p)^2
\end{equation}

\begin{equation}
J_{pm}=m_{pm}l^2_{pm}
\end{equation}

\begin{equation}
J_{arm}+J_{pend1}+J_{sens}=m_1l^2_1J_{pm}+J_{ps}=m_2l^2_2
\end{equation}

\begin{equation}
l_1=\sqrt{\frac{m_r \frac{l^2_r}{12}+m_r(\frac{1}{2}l_r+l_b-l_a)^2+m_bl^2_a+m_{sens}(l_a-l_b-l_r-\frac{1}{2}l_c)^2}{m_1}}
\end{equation}

\begin{equation}
l_2=\sqrt{\frac{m_{pm}l^2_{pm}+m_p(r_b+\frac{1}{2}l_p)^2}{m_2}}
\end{equation}

... Hier sollten wir gucken, ob wie die Umformungen evtl. anders vornehmen.

\begin{eqnarray}
Liste der Zahlenwerte
\end{eqnarray}

\begin{equation}
\end{equation}

\begin{equation}
\end{equation}

Swing-Up
\begin{equation}
U = n \cdot g \cdot sign(E-E_0)\dot{\theta}_2\cos(\theta_2)
\end{equation}

\begin{equation}
E = E_{pot}+E_{kin}
\end{equation}

\begin{eqnarray}
E_{pot} &=& m_2gl_2(\cos(\theta_2)-1) \nonumber \\
E_{kin} &=& \frac{1}{2}\dot{\theta}^2_2\hat{J}_2
\end{eqnarray}

\begin{equation}
\end{equation}
