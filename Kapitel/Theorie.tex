\section{Theorie}
\label{sec.Theorie}
In diesem Abschnitt wird auf die Theorie des sogenannten \emph{Furuta Pendels} eingegangen. Zunächst wird ein Modell des Pendels beschrieben. Im Anschluss wird die Theorie dieses Modells auf ein echtes Furuta Pendel angewandt. %Thomas: Diesen Text überarbeiten wenn der Abschnitt fertig ist.

\subsection{Furuta Pendel}
\label{sub.Furuta-Pendel}
Bei dem \emph{Furuta Pendel}, auch drehbares invertiertes Pendel genannt, handelt es sich um ein 1992 von Katsuhisa Furuta entwickeltes mechanisches Pendel. 
Es besteht aus einem angetriebenen Arm, welcher in der horizontalen Ebene rotieren kann. 
An diesem Arm ist ein Pendelarm befestigt, welcher frei in der zum angetriebenen Arm orthogonalen Ebene rotiert. 
Das Furuta Pendel ist ein Beispiel eines komplexen nichtlinearen Oszillators. \citep{Cazzolato.2011}

\begin{figure}[htbp]
	\centering
	\includegraphics[width=1.\textwidth]{Grafiken/furuta3.jpg}
	\caption{Furuta Pendel nach~\cite{Cazzolato.2011}. }
	\label{fig.furuta-schematic}
\end{figure}

Abbildung~\ref{fig.furuta-schematic} zeigt das Schema eines Furuta Pendels, welches von einem DC-Elektromotor angetrieben wird.
Mit dem Motor wird ein Drehmoment $\tau_1$ auf Arm~1 des Pendels gebracht. 
Das Drehmoment $\tau_2$ bezeichnet das Kopplungsmoment zwischen Arm~1 und Arm~2. 
Die Verbindung zwischen Arm~1 und Arm~2 rotiert frei und ist nicht angetrieben.
Die beiden Arme haben die Längen $L_1$ und $L_2$ sowie die Massen $m_1$ und $m_2$.
$l_1$ und $l_2$ bezeichnen die jeweiligen Abstände der Massenmittelpunkte zum Rotationspunkt der Arme.
Die Arme besitzen die Trägheitstensoren $J_1$ und $J_2$ um ihren jeweiligen Massenmittelpunkt und ihre Gelenke unterliegen linearer Dämpfung mit den Dämpfungskonstanten $b_1$ und~$b_2$.

\subsection{Annahmen}
\label{sub.sub.Annahmen}
Vor der Aufstellung des Gleichungssystems werden in~\cite{Cazzolato.2011} einige Annahmen getroffen, die an dieser Stelle kurz wiederholt werden:
\begin{enumerate}
\item Die Kopplung zwischen Motorachse und Arm~1 wird als ideal starr angenommen.
\item Die Motorachse und beide Pendelarme werden als ideal starr angenommen.
\item Die Koordinatensysteme von Arm~1 und Arm~2 sind die Hauptachsensysteme, sodass die Trägheitstensoren diagonal sind.
\item Die Trägheit des Motors wird vernachlässigt. %Thomas: trifft das auf uns zu????
\item Es wird von reiner linearer (viskoser) Dämpfung ausgegangen.
\end{enumerate}

\subsection{Mathematisches Modell}
\label{sub.sub.Mathematisches-Modell}
Mit den in Abschnitt~\ref{sub.sub.Annahmen} getroffenen Annahmen wird in~\cite{Cazzolato.2011} ein mathematisches Modell entwickelt, mit dem die Winkelbeschleunigung der beiden Pendelarme in Abhängigkeit der Zustands\-varibalen beschrieben werden kann. 
Zunächst wird die Gleichung für die beiden Drehmomente $\tau_1$ und $\tau_2$ in Abhängigkeit der Trägheitstensoren $J_1$ und $J_2$ aufgestellt.
(Für nähre Ausführungen sei auf~\cite{Cazzolato.2011} verwiesen.)

\begin{equation}
\label{eqn.TauComplex}
\begin{bmatrix}
\tau_1 \\
\tau_2
\end{bmatrix}
=
\begin{bmatrix}
\begin{pmatrix}
\ddot{\theta}_1(J_{1zz}+m_1l^2_1+m_2L^2_1+(J_{2yy}+m_2l^2_2) 						\\
\times \sin^2(\theta_2)+J_{2xx}cos^2(\theta_2))+\ddot{\theta}_2m_2L_1l_2\cos(\theta_2)			\\
-m_2L_1l_2\sin(\theta_2)\dot{\theta}^2_2+\dot{\theta}_1\dot{\theta}_2\sin(2\theta_2)	\\
\times(m_2l^2_2+J_{2yy}-J_{2xx})+b_1\dot{\theta}_1
\end{pmatrix}
\\
\begin{pmatrix}
\ddot{\theta}_1m_2L_1l_2\cos(\theta_2)+\ddot{\theta}_2(m_2l^2_2+J_{2zz})	\\
+\frac{1}{2}\dot{\theta}_1\sin(2\theta_2)(-m_2l^2_2-J_{2yy}+J_{2xx})						\\
+b_2\dot{\theta}_2+gm_2l_2\sin(\theta_2)
\end{pmatrix}
\end{bmatrix}
\end{equation}

Die Trägheitstensoren der beiden Pendelarme können nun unter den Annahmen $J_{xx} \approx 0$ und $J_{yy} \approx J_{zz}$ wie folgt vereinfacht werden: 
\begin{align}
\label{eqn.Tragheitstensoren}
J_1&=
\begin{bmatrix}
J_{1xx} & 0 & 0\\
0 & J_{1yy} & 0\\
0 & 0 & J_{1zz}
\end{bmatrix}
&\approx
\begin{bmatrix}
0 & 0 & 0\\
0 & \bar{J}_{1} & 0\\
0 & 0 & \bar{J}_{1}
\end{bmatrix}
\nonumber \\
J_2&=
\begin{bmatrix}
J_{2xx} & 0 & 0\\
0 & J_{2yy} & 0\\
0 & 0 & J_{2zz}
\end{bmatrix}
&\approx
\begin{bmatrix}
0 & 0 & 0\\
0 & \bar{J}_{2} & 0\\
0 & 0 & \bar{J}_{2}
\end{bmatrix}
\end{align}

Folgende Bezeichnungen werden zur Vereinfachung von Gleichung~\ref{eqn.TauComplex} eingeführt.
\begin{eqnarray}
\hat{J}_1 &=& \bar{J}_1 + m_1l^2_1	\nonumber	\\
\hat{J}_2 &=& \bar{J}_2 + m_2l^2_2	\nonumber	\\
\hat{J}_0 &=& \bar{J}_1 + m_1l^2_1 + m_2L^2_1
\end{eqnarray}

So lassen sich die Formeln für die beiden Drehmomente $\tau_1$ und $\tau_2$ darstellen als:
\begin{equation}
\begin{bmatrix}
\tau_1 \\
\tau_2
\end{bmatrix}
=
\begin{bmatrix}
\begin{pmatrix}
\ddot{\theta}_1(\hat{J}_0+\hat{J}_2\sin^2(\theta_2))+\ddot{\theta}_2m_2L_1l_2\cos(\theta_2)			\\
-m_2L_1l_2\sin(\theta_2)\dot{\theta}^2_2+\dot{\theta}_1\dot{\theta}_2\hat{J}_2\sin(2\theta_2)+b_1\dot{\theta}_1
\end{pmatrix}
\\
\begin{pmatrix}
\ddot{\theta}_1m_2L_1l_2\cos(\theta_2)+\ddot{\theta}_2\hat{J}_2-\frac{1}{2}\dot{\theta}^2_1\hat{J}_2\sin(2\theta_2)						\\
+b_2\dot{\theta}_2+gm_2l_2\sin(\theta_2)
\end{pmatrix}
\end{bmatrix}
\end{equation}

Hieraus ergeben sich die folgenden nichtlinearen Differentialgleichungen für die Bewegung des Pendels. 
In~\cite{Cazzolato.2011} wird bei der Aufstellung dieser ein Vorzeichenfehler gemacht, der an dieser Stelle korrigiert ist und auf den nicht näher eingegangen wird.


%\colorbox{yellow}{Die folgenden Gleichungen enthalten noch den Fehler, den wir finden sollten.} \\
%\colorbox{yellow}{Ich habe die Lösung gerade nicht parat.Thomas: ich auch noch nicht}
Nachfolgend die nichtlinearen Differentialgleichungen für die Winkelbeschleunigung von Arm~1;
\begin{multline}
\ddot{\theta}_1 =
\frac{
\begin{bmatrix}
	-\hat{J}_2b_1 \\ 
	m_2L_1l_2\cos(\theta_2)b_2 \\ 
	-\hat{J}^2_2\sin(2\theta_2) \\ 
	-\frac{1}{2}\hat{J}_2m_2L_1l_2\cos(\theta_2)\sin(2\theta_2) \\ 
	\hat{J}_2m_2L_1l_2\sin(\theta_2)
\end{bmatrix}^T
\begin{bmatrix}
	\dot{\theta}_1 \\ 
	\dot{\theta}_2 \\ 
	\dot{\theta}_1\dot{\theta}_2 \\ 
	\dot{\theta}^2_1 \\ 
	\dot{\theta}^2_2
\end{bmatrix} }
{\hat{J}_0\hat{J}_2+\hat{J}^2_2\sin^2(\theta_2)-m^2_2L^2_1l^2_2\cos^2(\theta_2)} \\ 
%<--linebreak here
+
\frac{
\begin{bmatrix}
	\hat{J}_2 \\ 
	-m_2L_1l_2\cos(\theta_2) \\ 
	\frac{1}{2}m^2_2l^2_2L_1\sin(2\theta_2)
\end{bmatrix}^T
\begin{bmatrix}
	\tau_1 \\ 
	\tau_2 \\ 
	g
\end{bmatrix} }
{\hat{J}_0\hat{J}_2+\hat{J}^2_2\sin^2(\theta_2)-m^2_2L^2_1l^2_2\cos^2(\theta_2)}
\end{multline}

und die Winkelbeschleunigung von Arm~2:

\begin{multline}
\ddot{\theta}_2 =
\frac{
\begin{bmatrix}
	m_2L_1l_2\cos(\theta_2)b_1 \\ 
	-b_2(\hat{J}_0+\hat{J}_2\sin^2(\theta_2)) \\ 
	m_2L_1l_2\hat{J}_2\cos(\theta_2)\sin(2\theta_2) \\ 
	\frac{1}{2}\sin(2\theta_2)(\hat{J}_0\hat{J}_2+\hat{J}^2_2\sin^2(\theta_2)) \\ 
	-\frac{1}{2}m^2_2L^2_1l^2_2\sin(2\theta_2)
\end{bmatrix}^T
\begin{bmatrix}
	\dot{\theta}_1 \\ 
	\dot{\theta}_2 \\ 
	\dot{\theta}_1\dot{\theta}_2 \\ 
	\dot{\theta}^2_1 \\ 
	\dot{\theta}^2_2
\end{bmatrix} }
{\hat{J}_0\hat{J}_2+\hat{J}^2_2\sin^2(\theta_2)-m^2_2L^2_1l^2_2\cos^2(\theta_2)} \\
%<--linebreak here
+
\frac{
\begin{bmatrix}
	-m_2l_2\cos(\theta_2) \\ 
	\hat{J}_0+\hat{J}_2\sin^2(\theta_2) \\ 
	-m_2l_2\sin(\theta_2)(\hat{J}_0+\hat{J}_2\sin^2(\theta_2))
\end{bmatrix}^T
\begin{bmatrix}
	\tau_1 \\ 
	\tau_2 \\ 
	g
\end{bmatrix} }
{\hat{J}_0\hat{J}_2+\hat{J}^2_2\sin^2(\theta_2)-m^2_2L^2_1l^2_2\cos^2(\theta_2)}
\end{multline}

Diese entsprechen den Gleichungen~(33) und~(34) in~\cite{Cazzolato.2011}.

\subsection{Linearisierung}
\label{sub.sub.Linearisierung}
Die korrigierten nichtlinearen Differentialgleichungen aus Abschnitt~\ref{sub.sub.Mathematisches-Modell} werden in diesem Abschnitt nach dem Vorbild von~\cite{Cazzolato.2011} für den grenz\-stabilen oberen Gleichgewichtszustand ($\theta_2 = \pi$) linearisiert.

So ergeben sich folgende Zustands-Variablen für den Gleichgewichtszustand:
\begin{eqnarray}
\theta_{1e} &=& 0		\nonumber \\
\theta_{2e} &=& \pi	\nonumber \\
\dot{\theta}_{1e} &=& 0		\nonumber \\
\dot{\theta}_{2e} &=& 0	
\end{eqnarray}

Nach dem Vorbild von~\cite{Cazzolato.2011} kann so ein lineares Differentialgleichungssystem erster Ordnung aufgestellt werden. 
Ein solches System ist notwendig, um gängige Theorien für den Entwurf von Reglern zu nutzen~\citep{Werner.2013}. 
\begin{equation}
\begin{bmatrix}
\dot{\theta}_1 \\ 
\dot{\theta}_2 \\ 
\ddot{\theta}_1 \\ 
\ddot{\theta}_2
\end{bmatrix}
=
\begin{bmatrix}
0 & 0 & 1 & 0 \\ 
0 & 0 & 0 & 1 \\ 
A_{31} & A_{32} & A_{33} & A_{34} \\ 
A_{41} & A_{42} & A_{43} & A_{44}
\end{bmatrix}
\begin{bmatrix}
\theta_1 \\ 
\theta_2 \\ 
\dot{\theta}_1 \\
\dot{\theta}_2
\end{bmatrix}
+
\begin{bmatrix}
0 & 0 \\ 
0 & 0 \\ 
B_{31} & B_{32} \\ 
B_{41} & B_{42}
\end{bmatrix}
\begin{bmatrix}
\tau_1 \\ 
\tau_2
\end{bmatrix}
\end{equation}

Nach dem Einsetzen der Zustands-Variablen im Gleichgewichtszustand erhalten wir die Matrixeinträge in Abhängigkeit der Variablen aus Abschnitt~\ref{sub.sub.Mathematisches-Modell}.
\begin{eqnarray}
A_{31} &=& 0	\nonumber \\
A_{32} &=& \frac{gm^2_2l^2_2L_1}{(\hat{J}_0\hat{J}_2-m^2_2L^2_1l^2_2)}	\nonumber \\
A_{33} &=& \frac{-b_1\hat{J}_2}{(\hat{J}_0\hat{J}_2-m^2_2L^2_1l^2_2)}	\nonumber \\
A_{34} &=& \frac{-b_2m_2l_2L_1}{(\hat{J}_0\hat{J}_2-m^2_2L^2_1l^2_2)}	\nonumber \\
A_{41} &=& 0	\nonumber \\
A_{42} &=& \frac{gm_2l_2\hat{J}_0}{(\hat{J}_0\hat{J}_2-m^2_2L^2_1l^2_2)}	\nonumber \\
A_{43} &=& \frac{-b_1m_2l_2L_1}{(\hat{J}_0\hat{J}_2-m^2_2L^2_1l^2_2)}	\nonumber \\
A_{44} &=& \frac{-b_2\hat{J}_0}{(\hat{J}_0\hat{J}_2-m^2_2L^2_1l^2_2)}	\nonumber \\
\end{eqnarray}
\begin{eqnarray}
B_{31} &=& \frac{\hat{J}_2}{(\hat{J}_0\hat{J}_2-m^2_2L^2_1l^2_2)}	\nonumber \\
B_{41} &=& \frac{m_2L_1l_2}{(\hat{J}_0\hat{J}_2-m^2_2L^2_1l^2_2)}	\nonumber \\
B_{32} &=& \frac{m_2L_1l_2}{(\hat{J}_0\hat{J}_2-m^2_2L^2_1l^2_2)}	\nonumber \\
B_{42} &=& \frac{\hat{J}_0}{(\hat{J}_0\hat{J}_2-m^2_2L^2_1l^2_2)}	\nonumber \\
\end{eqnarray}

