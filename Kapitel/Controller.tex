\section{Design der Controller}
\label{sec.Controller}
\subsection{Signalverarbeitung}
\label{signalverarbeitung} 
Das Ausgangssignal der Inkrementalgeber wird zunächst auf Radiant-Werte umgerechnet 

\subsection{Swing-Up-Controller}
\label{Swing-Up-Controller} 

Der Swing-Up-Controller soll dem Aufschwingen des Pendels dienen und im Bereich zwischen circa $20^\circ < \left| \theta_2 \right| < 90^\circ$  und basiert auf der Lyapunov-Funktion. Diese führt dazu, dass sich bei der Wahl von
$ \theta_2 = 0$ in der aufrechten Position des Pendels ein Regler ergibt, welche die Energie des Pendels minimiert:

\begin{equation}
U = n \cdot g \cdot sign(E-E_0)\dot{\theta}_2\cos(\theta_2)
\end{equation}

Der Term $E_0$ beschreibt die gewünschte, minimale Energie des Systems.Die Energie des Pendels $E$ setzt sich aus der potentiellen Energie $E_{pot}$ und der kinetischen Energie $E_{kin}$ zusammen. Die potentielle Energie lässt sich aufgrund der Referenz im höchsten Punkt des Pendels wie folgt berechnen:
\begin{equation}
E_{pot} = m_2 \cdot g \cdot l_2 \cdot (cos(\theta_2)-1)
\end{equation}

Die kinetische Energie ergibt sich zu:

\begin{equation}
E_{kin} = \frac{J_0}{2} \cdot \dot{\theta_2}^2
\end{equation}

Die Umsetzung des Reglers in Simulink ist in ABB % TODO

zu sehen. 

\begin{figure}[h!]
  \caption{Swing-Up-Controller}
  \centering
    %\includegraphics[width=0.5\textwidth]{swing_up}
\end{figure}


\subsection{Zweipunktregler}
\label{zweipunktregler} 

Für den Bereich zwischen circa $20^\circ < \left| \theta_2 \right| < 90^\circ$ soll der Zweipunktregler (eng. Bang-bang control) das invertierte Pendel regeln. Dieser schaltet die maximale positive oder negative Spannung auf den Motor, abhängig von $ \theta_2 $ und $ \dot{\theta_2} $, daraus folgt  Übertragungfunktion:

\begin{equation}
U = -10 \cdot sign(\theta_2 \cdot \cos(\dot{\theta}_2))
\end{equation}

  Das entsprechende Simulink-Modell in Abb. %TODO
zu sehen. 

\begin{figure}[h!]
  \caption{Swing-Up-Controller}
  \centering
   % \includegraphics[width=0.5\textwidth]{swing_up}
\end{figure}

\subsection{Catcher}
\label{catcher} 

Der Catcher dient der genaueren Regelung des Pendels nahe des oberen Equilibriums von $ \theta_2 $. Um in dem Bereich optimal und robust zu Regeln,wird ein LQ-Regler verwendet. Dieser basiert auf der Minimiering der Cost-Function, welche folgenden Form besitzt:

\begin{equation}
 V = \int_0^\infty \! (x^T(t) Qx(t) + u^t(t) R u(t))  \mathrm{d}t
\end{equation}

Erhöht man die Diagonalwerte der Q-Matrix, werden die Fehler stärker Gewichtet, allerdings steigt auch der Aufwand der Regelung. Die Matrix lässt sich über die Controllability-Matrix $C$ und den linken Eigenvektor q der Matrix $A$
wie folgt berechnen:

\begin{equation}
 Q = C' \cdot q'^T \cdot q^T \cdot C
\end{equation}

Für den Vektor $q$ wurden folgenden Werte angenommen:

\begin{equation}
q =\begin{bmatrix}
         180/\pi/180 \\
         0\\
         0\\
         180/\pi/0.1
        \end{bmatrix}
\end{equation}
 
Die Diagonalwerte der Matrix R der Cost-Funktion beeinflussen die Geschwindigkeit der Regelung und wurden zunächst für den Catcher auf eins gesetz.
Zusammen mit den Eingangs- und Ausgangsmatritzen A und B lässt sich mithilfe des Matlab-Befehls $F = -lqr(A,B,Q,R)$ der LQR-Controller generieren.
Das Simulink-Modell ist in Abb zu sehen.


\subsection{Stabilizer}
\label{stabilizer} 

Zur Rückführung des Pendels auf die Ausgangswinkel von $\theta_1$ und Beibehaltung der oberen Equilibrium-Position wird auch eine LQ-Regelung verwendet, welche sich nur durch die dreifache Einheitsmatrix für R und folgenden q-Vektor von dem Catcher unterscheidet:

\begin{equation}
q =\begin{bmatrix}
         180/\pi/1 \\
         0\\
         0\\
         180/\pi/0.1
        \end{bmatrix}
\end{equation}

